\documentclass{article}
\usepackage[utf8]{inputenc}
\usepackage[T1]{fontenc}
\usepackage{amsmath}
\usepackage{amsfonts}
\usepackage{amssymb}
\usepackage{enumitem}

\usepackage[margin=0.7in]{geometry} % Adjust the margins of document here.

\usepackage{tikz}                                          % Для простых рисунков в документе
\usetikzlibrary{matrix,arrows,decorations.pathmorphing,shapes.geometric,calc,snakes,backgrounds,arrows.meta}
\usepackage{xcolor}

\begin{document}
\pagestyle{plain}

\section*{Problems}

\subsection*{Second day — August 3, 1996}

\textbf{Problem 1.} (10 points)
Prove that if \( f : [0,1] \to [0,1] \) is a continuous function, then the sequence of iterates \( x_{n+1} = f(x_n) \) converges if and only if

\[
\lim_{n \to \infty} (x_{n+1} - x_n) = 0.
\]

\textbf{Problem 2.} (10 points)
Let \(\theta\) be a positive real number and let \(\cosh t = \frac{e^t + e^{-t}}{2}\) denote the hyperbolic cosine. Show that if \(k \in \mathbb{N}\) and both \(\cosh k\theta\) and \(\cosh (k + 1)\theta\) are rational, then so is \(\cosh \theta\).

\textbf{Problem 3.} (15 points)
Let \( G \) be the subgroup of \( GL_2(\mathbb{R}) \), generated by \( A \) and \( B \), where
\[
A = \begin{bmatrix}
2 & 0 \\
0 & 1
\end{bmatrix}, \quad B = \begin{bmatrix}
1 & 1 \\
0 & 1
\end{bmatrix}.
\]
Let \( H \) consist of those matrices \( \begin{bmatrix}
a_{11} & a_{12} \\
a_{21} & a_{22}
\end{bmatrix} \) in \( G \) for which \( a_{11}=a_{22}=1 \).

\begin{itemize}
    \item[(a)] Show that \( H \) is an abelian subgroup of \( G \).
    \item[(b)] Show that \( H \) is not finitely generated.
\end{itemize}

\textbf{Remarks.} \( GL_2(\mathbb{R}) \) denotes, as usual, the group (under matrix multiplication) of all \( 2 \times 2 \) invertible matrices with real entries (elements). \textit{Abelian} means commutative. A group is \textit{finitely generated} if there are a finite number of elements of the group such that every other element of the group can be obtained from these elements using the group operation.

\textbf{Problem 4.} (20 points) \\
Let \( B \) be a bounded closed convex symmetric (with respect to the origin)
set in \( \mathbb{R}^2 \) with boundary the curve \( \Gamma \). Let \( B \)
have the property that the ellipse of maximal area contained in \( B \)
is the disc \( D \) of radius 1 centered at the origin with boundary
the circle \( C \). Prove that \( A \cap \Gamma \neq \emptyset \)
for any arc \( A \) of \( C \) of length \( l(A) \geq \frac{\pi}{2} \).

\textbf{Problem 5.} (20 points)
\begin{itemize}
    \item[(i)] Prove that
    \[
    \lim_{x \to +\infty} \sum_{n=1}^{\infty} \frac{nx}{(n^2 + x)^2} = \frac{1}{2}.
    \]
    \item[(ii)] Prove that there is a positive constant \( c \) such that for every \( x \in [1, \infty) \) we have
    \[
    \left| \sum_{n=1}^{\infty} \frac{nx}{(n^2 + x)^2} - \frac{1}{2} \right| \leq \frac{c}{x}.
    \]
\end{itemize}

\textbf{Problem 6. (Carleman’s inequality)} (25 points)

(i) Prove that for every sequence \(\{a_n\}_{n=1}^\infty\), such that \(a_n > 0\), \(n = 1,2,\ldots\) and \(\sum_{n=1}^{\infty} a_n < \infty\), we have

\[
\sum_{n=1}^{\infty} (a_1 a_2 \ldots a_n)^{1/n} < e \sum_{n=1}^{\infty} a_n,
\]

where \( e \) is the natural log base.

(ii) Prove that for every \( \varepsilon > 0 \) there exists a sequence \(\{a_n\}_{n=1}^\infty\), such that \(a_n > 0\), \(n = 1,2,\ldots\), \(\sum_{n=1}^{\infty} a_n < \infty\) and

\[
\sum_{n=1}^{\infty} (a_1 a_2 \ldots a_n)^{1/n} > (e - \varepsilon) \sum_{n=1}^{\infty} a_n.
\]

\end{document}
