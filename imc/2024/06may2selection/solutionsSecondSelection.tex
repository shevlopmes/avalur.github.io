\documentclass{article}
\usepackage[utf8]{inputenc}
\usepackage[T1]{fontenc}
\usepackage{amsmath}
\usepackage{amsfonts}
\usepackage{amssymb}
\usepackage{enumitem}

\usepackage[margin=0.7in]{geometry} % Adjust the margins of document here.

\usepackage{tikz}                                          % Для простых рисунков в документе
\usetikzlibrary{matrix,arrows,decorations.pathmorphing,shapes.geometric,calc,snakes,backgrounds,arrows.meta}
\usepackage{xcolor}

\begin{document}
\pagestyle{plain}

\section*{Problems and solutions}

\subsection*{Second Selection Test, May 2024}


\textbf{Problem 1.} (10 points)
For any integer $n \geq 2$ and two $n \times n$ matrices with real entries $A, B$ that satisfy the equation
\[
A^{-1} + B^{-1} = (A + B)^{-1},
\]
prove that $\det(A) = \det(B)$. Does the same conclusion follow for matrices with complex entries?

\textbf{Solution.} Multiplying the equation by $(A + B)$ we get
\[
I = (A + B)(A^{-1} + B^{-1}) = (A + B)(A + B)^{-1} = AA^{-1} + AB^{-1} + BA^{-1} + BB^{-1} = I + AB^{-1} + BA^{-1} + I = 0.
\]
Let $X = AB^{-1}$, then $A = XB$ and $BA^{-1} = X^{-1}$, so we have $X + X^{-1} + I = 0$; multiplying by $(X - I)X$,
\[
0 = (X - I)X \cdot (X + X^{-1} + I) = (X - I) \cdot (X^2 + X + I) = X^3 - I.
\]

Hence,
\[
X^3 = I \quad \text{so} \quad (\det X)^3 = \det(X^3) = \det I = 1 \quad \text{thus} \quad \det X = 1
\]
\[
\det A = \det(XB) = \det X \cdot \det B = \det B.
\]
In case of complex matrices the statement is false. Let $\omega = \frac{1}{2}(-1 + i\sqrt{3})$. Obviously $\omega \notin \mathbb{R}$ and $\omega^3 = 1$, so $0 = 1 + \omega + \omega^2 = 1 + \omega + \overline{\omega}$.

Let $A = I$ and let $B$ be a diagonal matrix with all entries along the diagonal equal to either $\omega$ or $\overline{\omega} = \omega^2$ such a way that $\det(B) \neq 1$ (if $n$ is not divisible by 3 then one may set $B = \omega I$). Then $A^{-1} = I$, $B^{-1} = B$. Obviously $I + B + B = 0$ and
\[
(A + B)^{-1} = (-B)^{-1} = -B = I + B = A^{-1} + B^{-1}.
\]
By the choice of $A$ and $B$, $\det A = 1 \neq \det B$.

\textbf{Problem 2.} (10 points)
Let \( f : [0; +\infty) \to \mathbb{R} \) be a continuous function such that \( \lim_{x \to +\infty} f(x) = L \) exists (it may be finite or infinite). Prove that
\[
\lim_{n \to \infty} \int_0^1 f(nx) \, dx = L.
\]

\textbf{Solution 1.} \textit{Case 1: $L$ is finite.}
Take an arbitrary $\epsilon > 0$.
We construct a number $K \geq 0$ such that for every $n \geq K$ we have
\[
\left| \int_0^1 f(nx) \, dx - L \right| < \epsilon.
\]
Since $\lim_{x \to \infty} f(x) = L$, there exists a $K_1 \geq 0$ such that $|f(x) - L| < \frac{\epsilon}{2}$ for every $x \geq K_1$. Hence, for $n \geq K_1$ we have
\[
\left| \int_0^1 f(nx) \, dx - L \right| = \left| \frac{1}{n} \int_0^n f(x) \, dx - L \right| = \frac{1}{n} \left| \int_0^n (f(x) - L) \, dx \right| \leq
\]
\[
\leq \frac{1}{n} \left( \int_0^{K_1} |f(x) - L| \, dx + \int_{K_1}^n |f(x) - L| \, dx \right) \leq \frac{1}{n} \left( \int_0^{K_1} |f(x) - L| \, dx + (n - K_1) \frac{\epsilon}{2} \right) =
\]
\[
= \frac{1}{n} \left( \int_0^{K_1} |f(x) - L| \, dx\right) + \frac{\epsilon}{2}
\]

If $n \geq K_2 = \frac{2}{\epsilon} \int_0^{K_1} |f - L| \, dx$ then the first term is at most $\frac{\epsilon}{2}$. Then for $x \geq K := \max(K_1, K_2)$, we have
\[
\left| \int_0^1 f(nx) \, dx - L \right| < \frac{\epsilon}{2} + \frac{\epsilon}{2} = \epsilon.
\]


\textbf{Case 2:} $L = +\infty$. Take an arbitrary real $M$; we need a $K \geq 0$ such that
\[
\int_0^1 f(nx) \, dx > M
\]
for every $x \geq K$. Since $\lim_{x \to \infty} f(x) = \infty$, there exists a $K_1 \geq 0$ such that $f(x) > M + 1$ for every $x \geq K_1$. Hence, for $n \geq 2K_1$ we have
\[
\int_0^1 f(nx) \, dx = \frac{1}{n} \int_0^n f(x) \, dx = \frac{1}{n} \left( \int_0^{K_1} f(x) \, dx + \int_{K_1}^n f(x) \, dx \right) =
\]
\[
= \frac{1}{n} \left( \int_0^{K_1} f(x) \, dx + \int_{K_1}^n (M + 1) \, dx \right) = \frac{1}{n} \left( \int_0^{K_1} f \, dx + (n - K_1)(M + 1) \right).
\]
If $n \geq K_2 := \frac{1}{\int_0^{K_1} f - K_1(M + 1)}$, then the first term is at most $-\frac{1}{2}$. For $x \geq K := \max(K_1, K_2)$, we have
\[
\int_0^1 f(nx) \, dx > M.
\]

\textbf{Case 3:} $L = -\infty$. We can repeat the steps in Case 2 for the function $-f$.


\textbf{Solution 2.} Let $F(x) = \int_0^x f$. For $t > 0$ we have
\[
\int_0^1 f(tx) \, dx = \frac{F(t)}{t}.
\]
Since $\lim_{t \to \infty} t = \infty$ in the denominator and $\lim_{t \to \infty} F'(t) = \lim_{t \to \infty} f(t) = L$, L'Hospital's rule proves
\[
\lim_{t \to \infty} \frac{F(t)}{t} = \lim_{t \to \infty} \frac{F'(t)}{1} = \lim_{t \to \infty} f(t) = L.
\]
Then it follows that
\[
\lim_{n \to \infty} \frac{F(n)}{n} = L.
\]

\textbf{Problem 3.} (10 points)
For a positive integer \( n \),
let \( f(n) \) be the number obtained by writing \( n \) in binary and replacing every \( 0 \) with \( 1 \) and vice versa. For example, \( n = 23 \) is \( 10111 \) in binary, so \( f(23) \) is \( 1000 \) in binary, therefore \( f(23) = 8 \). Prove that
\[
\sum_{k=1}^n f(k) \leq \frac{n^2}{4}.
\]
When does equality hold?


\textbf{Solution.}
If \( r \) and \( k \) are positive integers with \( 2^{r-1}
\leq k < 2^r \) then \( k \) has \( r \) binary digits,
so \( k + f(k) = 11\ldots1_2 = 2^r - 1 \). Assume that \( 2^{s-1} - 1 \leq n \leq 2^s - 1 \). Then
\[
\frac{n(n+1)}{2} + \sum_{k=1}^n f(k) = \sum_{k=1}^n (k + f(k)) =
\]
\[
= \sum_{r=1}^{s-1} \sum_{k=2^{r-1}}^{2^r-1} (k + f(k)) + \sum_{k=2^{s-1}}^n (k + f(k)) =
\]
\[
= \sum_{r=1}^{s-1} 2^{2r-1} \cdot (2^r - 1) + (n - 2^{s-1} + 1) \cdot (2^s - 1) =
\]
\[
= \sum_{r=1}^{s-1} \left(2^{2r-1} - \sum_{r=1}^{s-1} 2^r + 1\right) + (n - 2^{s-1} + 1)(2^s - 1) =
\]
\[
= \frac{2}{3} (4^{s-1} - 1) - (2^{s-1} - 1) + (2^s - 1)n - \frac{1}{3}4^s + 2^s - 2
\]
\[
= (2^s - 1)n - \frac{1}{3}4^s + 2^s - \frac{2}{3}
\]

and therefore
\[
\frac{n^2}{4} - \sum_{k=1}^n f(k) = \frac{n^2}{4} - \left( (2^s-1)n - \frac{1}{3}4^s + 2^s - \frac{2}{3} - \frac{n(n+1)}{2} \right) =
\]
\[
= \frac{3}{4}n^2 - (2^s - \frac{3}{2})n + \frac{1}{3}4^s - 2^s + \frac{2}{3} =
\]
\[
= \frac{3}{4} \left(n - \frac{2^{s+1} - 2}{3} \right) \left(n - \frac{2^{s+1} - 4}{3} \right).
\]
Notice that the difference of the last two factors is less than 1, and one of them must be an integer: $\frac{2^{s+1} - 2}{3}$ is integer if \( s \) is even, and $\frac{2^{s+1} - 4}{3}$ is integer if \( s \) is odd. Therefore, either one of them is 0, resulting a zero product, or both factors have the same sign, so the product is strictly positive. This solves the problem and shows that equality occurs if \( n = \frac{2^{s+1} - 2}{3} \) (s is even) or \( n = \frac{2^{s+1} - 4}{3} \) (s is odd).

\textbf{Problem 4.} (10 points)
For any positive integer \(m\), denote by \(P(m)\) the product of positive divisors
of \(m\) (e.g. \(P(6) = 36\)).
For every positive integer \(n\) define the sequence
\[
a_1(n) = n, \quad a_{k+1}(n) = P(a_k(n)) \quad (k = 1, 2, \ldots, 2024).
\]
Determine whether for every set \(S \subseteq \{1, 2, \ldots, 2025\}\),
there exists a positive integer \(n\) such that the following condition is satisfied:
\[
\textit{For every } k \textit{ with } 1 \leq k \leq 2025,
\textit{ the number } a_k(n)
\textit{ is a perfect square if and only if } k \in S.
\]

\textbf{Solution.} We prove that the answer is yes;
for every $S \subseteq \{1, 2, \ldots, 2025\}$ there exists a suitable $n$.
Specially, $n$ can be a power of 2: $n = 2^{w_1}$ with some nonnegative integer $w_1$. Write $a_k(n) = 2^{w_k}$; then
\[
2^{w_{k+1}} = a_{k+1}(n) = P(a_k(n)) = P(2^{w_k}) = 1 \cdot 2 \cdot 4 \cdot \ldots \cdot 2^{w_k} = 2^{\frac{w_k(w_k+1)}{2}},
\]
so
\[
w_{k+1} = \frac{w_k(w_k + 1)}{2}.
\]
The proof will be completed if we prove that for each choice of $S$ there exists an initial value $w_1$
such that $w_k$ is even if and only if $k \in S$.

\textbf{Lemma.} Suppose that the sequences $(b_1, b_2, \ldots)$ and $(c_1, c_2, \ldots)$
satisfy $b_{k+1} = \frac{b_k(b_k+1)}{2}$ and $c_{k+1} = c_k(c_k+1)$ for $k \geq 1$, and $c_1 = b_1 + 2^m$.
Then for each $k = 1, \ldots m$ we have $c_k \equiv b_k \pmod{2^m-k+2}$.

As an immediate corollary, we have $b_k \equiv c_k \pmod{2}$ for $1 \leq k \leq m$ and $b_{m+1} \equiv c_{m+1} + 1 \pmod{2}$.

\textbf{Proof.} We prove the lemma by induction.
For $k = 1$ we have $c_1 = b_1 + 2^m$ so the statement holds.
Suppose the statement is true for some $k < m$, then for $k + 1$ we have
\[
c_{k+1} = \frac{c_k (c_k + 1)}{2} = \frac{(b_k + 2^m-k+1)(b_k + 2^m-k+1 + 1)}{2}
\]
\[
= \frac{b_k^2 + 2^{m-k}+2b_k + 2^{2m-2k}+2 + b_k + 2^{m-k}+1}{2}
\]
\[
= \frac{b_k(b_k + 1) + 2^{m-k} + 2^{m-k+1}b_k + 2^{2m-2k+1}}{2}
\]
\[
\equiv b_k(b_k + 1) + 2^{m-k} \pmod{2^{m-(k+1)+2}},
\]
therefore $c_{k+1} \equiv b_{k+1} + 2^{m-(k+1)+1} \pmod{2^{m-(k+1)+2}}$.


Going back to the solution of the problem,
for every $1 \leq m \leq 2025$ we construct inductively
a sequence $(v_1, v_2, \dots)$ such that $v_{k+1} = \frac{v_k(v_k+1)}{2}$,
and for every $1 \leq k \leq m$, $v_k$ is even
if and only if $k \in S$.

For $m = 1$ we can choose $v_1 = 0$ if $1 \in S$ or $v_1 = 1$ if $1 \notin S$.
If we already have such a sequence $(v_1, v_2, \dots)$
for a positive integer $m$, we can choose either the same sequence
or choose $v_1' = v_1 + 2^m$ and apply the same recurrence $v_{k+1}' = \frac{v_k'(v_k'+1)}{2}$. By the Lemma, we have $v_k \equiv v_k' \pmod{2}$ for $k \leq m$, but $v_{m+1}$ and $v_{m+1}'$ have opposite parities; hence, either the sequence $(v_k)$ or the sequence $(v_k')$ satisfies the condition for $m + 1$.

Repeating this process for $m = 1, 2, \dots, 2025$, we obtain a suitable sequence $(v_k)$.

\textbf{Problem 5.} (10 points)
Determine whether or not there exist 15 integers
\( m_1, \dots, m_{15} \) such that
\[
\sum_{k=1}^{15} m_k \cdot \arctan(k) = \arctan(16).
\]

\textbf{Solution}
We show that such integers \(m_1, \dots, m_{15}\) do not exist.
Suppose that equation is satisfied by some integers
\(m_1, \dots, m_{15}\). Then the argument of the complex number \(z_1 = 1 + 16i\) coincides with the argument of the complex number
\[
z_2 = (1 + i)^{m_1} (1 + 2i)^{m_2} (1 + 3i)^{m_3} \cdots (1 + 15i)^{m_{15}}.
\]
Therefore the ratio \( R = \frac{z_2}{z_1} \) is real (and not zero). As \(\operatorname{Re}(z_1) = 1\) and \(\operatorname{Re}(z_2)\) is an integer, \(R\) is a nonzero integer.

By considering the squares of the absolute values of \(z_1\) and \(z_2\), we get
\[
(1 + 16^2)R^2 = \prod_{k=1}^{15} (1 + k^2)^{m_k}.
\]
Notice that \( p = 1 + 16^2 = 257 \)
is a prime (the fourth Fermat prime), which yields an easy contradiction
through \(p\)-adic valuations: all prime factors in the right hand side are strictly
below \(p\) (as \(k < 16\) implies \(1 + k^2 < p\)).
On the other hand, in the left hand side the prime \(p\)
occurs with an odd exponent.

\end{document}
