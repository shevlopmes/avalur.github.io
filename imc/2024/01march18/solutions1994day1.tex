\documentclass{article}
\usepackage[utf8]{inputenc}
\usepackage[T1]{fontenc}
\usepackage{amsmath}
\usepackage{amsfonts}
\usepackage{amssymb}
\usepackage{enumitem}

\usepackage[margin=0.7in]{geometry} % Adjust the margins of document here.

\usepackage{tikz}                                          % Для простых рисунков в документе
\usetikzlibrary{matrix,arrows,decorations.pathmorphing,shapes.geometric,calc,snakes,backgrounds,arrows.meta}
\usepackage{xcolor}

\begin{document}
\pagestyle{plain}

\section*{Problems and solutions}

\subsection*{First day — July 29, 1994}

\paragraph{Problem 1.} (13 points)
\begin{enumerate}
    \item[(a)] Let $A$ be a $n \times n$ ($n \geq 2$) symmetric invertible matrix with real positive elements. Show that $z_n \leq n^2 - 2n$ where $z_n$ is the number of zero elements in $A^{-1}$.
    \item[(b)] How many zero elements are there in the inverse of the $n \times n$ matrix
    \[
    A = \begin{pmatrix}
    1 & 1 & 1 & 1 & \dots & 1 \\
    1 & 2 & 2 & 2 & \dots & 2 \\
    1 & 2 & 1 & 1 & \dots & 1 \\
    1 & 2 & 1 & 2 & \dots & 2 \\
    \vdots & \vdots & \vdots & \vdots & \ddots & \vdots \\
    1 & 2 & 1 & 2 & \dots & \ddots
    \end{pmatrix}
    \]
\end{enumerate}
\textbf{Solution.} Denote by $a_{ij}$ and $b_{ij}$ the elements of $A$ and $A^{-1}$ respectively. Then for $k \neq m$ we have $\sum_{i=0}^n a_{ki} b_{im} = 0$ and from the positivity of $a_{ij}$ we conclude that at least one of $\{b_{im} : i = 1, 2, \ldots, n\}$ is positive and at least one is negative. Hence we have at least two non-zero elements in every column of $A^{-1}$. This proves part (a). For part (b), all $b_{ij}$ are zero except $b_{11} = 2$, $b_{nn} = (-1)^n$, $b_{ii+1} = b_{i+1i} = (-1)^i$ for $i = 1, 2, \ldots, n - 1$.

\paragraph{Problem 2.} (13 points)
Let $f \in C^1(a, b)$, $\lim\limits_{x \to a+} f(x) = +\infty$,
$\lim\limits_{x \to b-} f(x) = -\infty$
and $f'(x) + f^2(x) \geq -1$ for $x \in (a, b)$. Prove that $b - a \geq \pi$ and give an example where $b - a = \pi$.

\textbf{Solution.} From the inequality we get $\frac{d}{dx}(\arctan f(x) + x) = \frac{1}{1 + f^2(x)} \geq 0$ for $x \in (a, b)$. Thus $\arctan f(x) + x$ is non-decreasing in the interval and using the limits we get $\frac{\pi}{2} + a \leq -\frac{\pi}{2} + b$. Hence $b - a \geq \pi$. One has equality for $f(x) = \cot x$, $a = 0$, $b = \pi$.

\paragraph{Problem 3.} (13 points) Given a set $S$ of $2n - 1$ ($n \in \mathbb{N}$) different irrational numbers. Prove that there are $n$ different elements $x_1, x_2, \ldots, x_n \in S$ such that for all non-negative rational numbers $a_1, a_2, \ldots, a_n$ with $a_1 + a_2 + \cdots + a_n > 0$ we have that $a_1x_1 + a_2x_2 + \cdots + a_nx_n$ is an irrational number.

\textbf{Solution.} Let $\mathbb{I}$ be the set of irrational numbers,
$\mathbb{Q}$ – the set of rational numbers,
$\mathbb{Q}_+ = \mathbb{Q} \cap [0, \infty)$. We work by induction. For $n = 1$ the statement is trivial.
Let it be true for $n - 1$. We start to prove it for $n$.
From the induction argument, there are $n-1$
different elements $x_1, x_2, \ldots, x_{n-1} \in S$ such that
\begin{equation}
a_1x_1 + a_2x_2 + \cdots + a_{n-1}x_{n-1} \in \mathbb{I}
\end{equation}
for all $a_1, a_2, \ldots, a_n \in \mathbb{Q}^+$ with $a_1 + a_2 + \cdots + a_{n-1} > 0$.

Denote the other elements of $S$ by $x_n, x_{n+1}, \ldots, x_{2n-1}$. Assume the statement is not true for $n$. Then for $k = 0, 1, \ldots, n - 1$ there are $r_k \in \mathbb{Q}$ such that
\begin{equation}
\sum_{i=1}^{n-1} b_{ik}x_i + c_kx_{n+k} = r_k \text{ for some } b_{ik}, c_k \in \mathbb{Q}^+, \sum_{i=1}^{n-1} b_{ik} + c_k > 0.
\end{equation}

Also
\begin{equation}
\sum_{k=0}^{n-1} d_kx_{n+k} = R \text{ for some } d_k \in \mathbb{Q}^+, \sum_{k=0}^{n-1} d_k > 0, R \in \mathbb{Q}.
\end{equation}

If in (2) $c_k = 0$ then (2) contradicts (1). Thus $c_k \neq 0$ and without loss of generality one may take $c_k = 1$. In (2) also $\sum_{i=1}^{n-1} b_{ik} > 0$ in view of $x_{n+k} \in \mathbb{I}$. Replacing (2) in (3) we get
\begin{equation}
\sum_{k=0}^{n-1} d_k \left( -\sum_{i=1}^{n-1} b_{ik}x_i + r_k \right) = R \text{ or } \sum_{i=1}^{n-1} \left( \sum_{k=0}^{n-1} d_kb_{ik} \right) x_i \in \mathbb{Q},
\end{equation}
which contradicts (1) because of the conditions on $b$'s and $d$'s.

\textbf{Problem 4.} (18 points) Let $\alpha \in \mathbb{R} \setminus \{0\}$ and suppose that $F$ and $G$ are linear maps (operators) from $\mathbb{R}^n$ into $\mathbb{R}^n$ satisfying $F \circ G - G \circ F = \alpha F$.
\begin{enumerate}
    \item[(a)] Show that for all $k \in \mathbb{N}$ one has $F^k \circ G - G \circ F^k = \alpha k F^k$.
    \item[(b)] Show that there exists $k \geq 1$ such that $F^k = 0$.
\end{enumerate}

\textbf{Solution.} For a) using the assumptions we have
\begin{align*}
F^k \circ G - G \circ F^k &= \sum_{i=1}^{k} \left( F^{k-i+1} \circ G \circ F^{i-1} - F^{k-i} \circ G \circ F^i \right) \\
&= \sum_{i=1}^{k} F^{k-i} \circ (F \circ G - G \circ F) \circ F^{i-1} \\
&= \sum_{i=1}^{k} F^{k-i} \circ \alpha F \circ F^{i-1} = \alpha^k F^k.
\end{align*}

b) Consider the linear operator $L(F) = F \circ G - G \circ F$ acting over all $n \times n$ matrices $F$. It may have at most $n^2$ different eigenvalues. Assuming that $F^k \neq 0$ for every $k$ we get that $L$ has infinitely many different eigenvalues $\alpha^k$ in view of a) -- a contradiction.

\textbf{Problem 5.} (18 points)
\begin{enumerate}
    \item[a)] Let $f \in C[0, b]$, $g \in C(\mathbb{R})$ and let $g$ be periodic with period $b$. Prove that
    \[
    \lim_{n \to \infty} \int_{0}^{b} f(x)g(nx) \, dx = \frac{1}{b} \int_{0}^{b} f(x) \, dx \int_{0}^{b} g(x) \, dx.
    \]

    \item[b)] Find
    \[
    \lim_{n \to \infty} \int_{0}^{\pi} \frac{\sin x}{1 + 3\cos 2nx} \, dx.
    \]
\end{enumerate}

\textbf{Solution.} Set $\|g\|_1 = \int_{0}^{b} |g(x)| \, dx$ and
\[
\omega(f, t) = \sup \{|f(x) - f(y)| : x, y \in [0, b], |x - y| \leq t\}.
\]
In view of the uniform continuity of $f$ we have $\omega(f, t) \to 0$ as $t \to 0$. Using the periodicity of $g$ we get
\begin{align*}
\int_{0}^{b} f(x)g(nx) \, dx &= \sum_{k=1}^{n} \int_{\frac{b(k-1)}{n}}^{\frac{bk}{n}} f(x)g(nx) \, dx \\
&= \sum_{k=1}^{n} \int_{\frac{b(k-1)}{n}}^{\frac{bk}{n}} g(nx) \, dx + \sum_{k=1}^{n} \int_{\frac{b(k-1)}{n}}^{\frac{bk}{n}} \left( f(x) - f\left(\frac{bk}{n}\right) \right) g(nx) \, dx \\
&= \frac{1}{n} \sum_{k=1}^{n} \int_{0}^{b} g(x) \, dx + O(\omega(f, b/n)\|g\|_1)
\end{align*}

\[
= \frac{1}{b} \sum_{k=1}^{n} \left( \int_{\frac{b(k-1)}{n}}^{\frac{bk}{n}} f(x) \, dx \right) \left( \int_{0}^{b} g(x) \, dx \right) + \frac{1}{b} \sum_{k=1}^{n} \left( \int_{\frac{b(k-1)}{n}}^{\frac{bk}{n}} - \int_{\frac{b(k-1)}{n}}^{\frac{bk}{n}} f(x) \, dx \right) \left( \int_{0}^{b} g(x) \, dx \right) + O(\omega(f, b/n)\|g\|_1)
\]
\[
= \frac{1}{b} \int_{0}^{b} f(x) \, dx \int_{0}^{b} g(x) \, dx + O(\omega(f, b/n)\|g\|_1).
\]

This proves a). For b) we set $b = \pi$, $f(x) = \sin x$, $g(x) = (1 + 3\cos 2x)^{-1}$. From a) and
\[
\int_{0}^{\pi} \sin x \, dx = 2, \quad \int_{0}^{\pi} (1 + 3\cos 2x)^{-1} \, dx = \frac{\pi}{2}
\]
we get
\[
\lim_{n \to \infty} \int_{0}^{\pi} \frac{\sin x}{1 + 3\cos 2nx} \, dx = 1.
\]

\textbf{Problem 6.} (25 points) Let $f \in C^2[0, N]$ and $|f''(x)| < 1$, $f'(x) > 0$ for every $x \in [0, N]$. Let $0 \leq m_0 < m_1 < \ldots < m_k \leq N$ be integers such that $n_i = f(m_i)$ are also integers for $i = 0, 1, \ldots, k$. Denote $b_i = n_i - n_{i-1}$ and $a_i = m_i - m_{i-1}$ for $i = 1, 2, \ldots, k$.
\begin{enumerate}
    \item[a)] Prove that
    \[
    -1 < \frac{b_1}{a_1} < \frac{b_2}{a_2} < \ldots < \frac{b_k}{a_k} < 1.
    \]

    \item[b)] Prove that for every choice of $\Lambda > 1$ there are no more than $N/\Lambda$ indices $j$ such that $a_j > \Lambda$.

    \item[c)] Prove that $k \leq 3N^{2/3}$ (i.e. there are no more than $3N^{2/3}$ integer points on the curve $y = f(x)$, $x \in [0, N]$).
\end{enumerate}

\textbf{Solution.}
\begin{enumerate}
    \item[a)] For $i = 1, 2, \ldots, k$ we have
    \[
        b_i = f(m_i) - f(m_{i-1}) = (m_i - m_{i-1})f'(x_i)
\]
    for some $x_i \in (m_{i-1}, m_i)$.
    Hence $\frac{b_i}{a_i} = f'(x_i)$ and so $-1 < \frac{b_i}{a_i} < 1$.
    From the convexity of $f$ we have that $f'$ is increasing and
    $\frac{b_i}{a_i} = f'(x_i) < f'(x_{i+1}) = \frac{b_{i+1}}{a_{i+1}}$
    because of $x_i < m_i < x_{i+1}$.

    \item[b)] Set $S_{\Lambda} = \{j \in \{0,1,\ldots,k\} : a_j > \Lambda\}$. Then
    \[
    N \geq m_k - m_0 = \sum_{i=1}^{k} a_i \geq \sum_{j \in S_{\Lambda}} a_j > \Lambda |S_{\Lambda}|
    \]
    and hence $|S_{\Lambda}| < N/\Lambda$.

    \item[c)] All different fractions in $(-1,1)$ with denominators less or equal to $\Lambda$ are no more than $2\Lambda^2$. Using b) we get $k < N/\Lambda + 2\Lambda^2$. Put $\Lambda = N^{1/3}$ in the above estimate and get $k < 3N^{2/3}$.
\end{enumerate}

\end{document}
