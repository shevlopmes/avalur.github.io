\documentclass{article}
\usepackage[utf8]{inputenc}
\usepackage[T1]{fontenc}
\usepackage{amsmath}
\usepackage{amsfonts}
\usepackage{amssymb}
\usepackage{enumitem}

\usepackage[margin=0.7in]{geometry} % Adjust the margins of document here.

\usepackage{tikz}                                          % Для простых рисунков в документе
\usetikzlibrary{matrix,arrows,decorations.pathmorphing,shapes.geometric,calc,snakes,backgrounds,arrows.meta}
\usepackage{xcolor}

\begin{document}
\pagestyle{plain}

\section*{Problems and solutions}

\subsection*{Second day — August 3, 1996}

\textbf{Problem 1.} (10 points)
Prove that if \( f : [0,1] \to [0,1] \) is a continuous function, then the sequence of iterates \( x_{n+1} = f(x_n) \) converges if and only if

\[
\lim_{n \to \infty} (x_{n+1} - x_n) = 0.
\]

\textbf{Solution.} The "only if" part is obvious.
Now suppose that \( \lim_{n \to \infty} (x_{n+1} - x_n) = 0 \) and
the sequence \( \{x_n\} \) does not converge.
Then there are two cluster points \( K < L \). There must be points
from the interval \( (K, L) \) in the sequence.
There is an \( x \in (K, L) \) such that \( f(x) \neq x \).
Put \( \varepsilon = \frac{|f(x) - x|}{2} > 0 \).
Then from the continuity of the function \( f \) we get that
for some \( \delta > 0 \) for all \( y \in (x - \delta, x + \delta) \)
it is \( |f(y) - y| > \varepsilon \). On the other hand for \( n \)
large enough it is \( |x_{n+1} - x_n| < 2\delta \)
and \( |f(x_n) - x_n| = |x_{n+1} - x_n| < \varepsilon \).
So the sequence cannot come into the interval \( (x - \delta, x + \delta) \),
but also cannot jump over this interval.
Then all cluster points have to be at most \( x - \delta \)
(a contradiction with \( L \) being a cluster point),
or at least \( x + \delta \) (a contradiction with \( K \) being a cluster point).

\textbf{Problem 2.} (10 points)
Let \(\theta\) be a positive real number and let \(\cosh t = \frac{e^t + e^{-t}}{2}\) denote the hyperbolic cosine. Show that if \(k \in \mathbb{N}\) and both \(\cosh k\theta\) and \(\cosh (k + 1)\theta\) are rational, then so is \(\cosh \theta\).

\textbf{Solution.} First we show that

\[
\text{(1)} \quad \text{If } \cosh t \text{ is rational and } m \in \mathbb{N}, \text{ then } \cosh mt \text{ is rational.}
\]

Since \(\cosh 0.t = \cosh 0 = 1 \in \mathbb{Q}\) and \(\cosh 1.t = \cosh t \in \mathbb{Q}\), (1) follows inductively from

\[
\cosh (m + 1)t = 2\cosh t.\cosh mt - \cosh (m - 1)t.
\]

The statement of the problem is obvious for \(k = 1\), so we consider \(k \geq 2\). For any \(m\) we have

\[
\text{(2)} \quad \cosh \theta = \cosh ((m + 1)\theta - m\theta) = \cosh (m + 1)\theta.\cosh m\theta - \sinh (m + 1)\theta.\sinh m\theta = \cosh (m + 1)\theta.\cosh m\theta - \sqrt{\cosh^2(m + 1)\theta - 1}.\sqrt{\cosh^2 m\theta - 1}
\]

Set \(\cosh k\theta = a\), \(\cosh (k + 1)\theta = b\), \(a, b \in \mathbb{Q}\). Then (2) with \(m = k\) gives

\[
\cosh \theta = ab - \sqrt{a^2 - 1}\sqrt{b^2 - 1}
\]
and then
\[
    (3) \quad (a^2 - 1)(b^2 - 1) = (ab - \cosh \theta)^2
= a^2b^2 - 2ab\cosh \theta + \cosh^2 \theta.
\]

Set \( \cosh (k^2 - 1)\theta = A, \cosh k^2 \theta = B\).
From (1) with \( m = k - 1 \) and \(t = (k + 1)\theta \)
we have \( A \in \mathbb{Q}\).
From (1) with \( m = k \) and \(t = k\theta\) we have \( B \in \mathbb{Q}\).
Moreover \( k^2 - 1 > k \) implies \( A > a \) and \( B > b\).
Thus \( AB > ab\). From (2) with \( m = k^2 - 1 \) we have
\[
    (4) \quad (A^2 - 1)(B^2 - 1) = (AB - \cosh \theta)^2
= A^2B^2 - 2AB\cosh \theta + \cosh^2 \theta.
\]

So after we cancel the \( \cosh 2\theta \) from (3) and (4)
we have a non-trivial linear equation in \( \cosh \theta \)
with rational coefficients.

\textbf{Problem 3.} (15 points)
Let \( G \) be the subgroup of \( GL_2(\mathbb{R}) \), generated by \( A \) and \( B \), where
\[
A = \begin{bmatrix}
2 & 0 \\
0 & 1
\end{bmatrix}, \quad B = \begin{bmatrix}
1 & 1 \\
0 & 1
\end{bmatrix}.
\]
Let \( H \) consist of those matrices \( \begin{bmatrix}
a_{11} & a_{12} \\
a_{21} & a_{22}
\end{bmatrix} \) in \( G \) for which \( a_{11}=a_{22}=1 \).

\begin{itemize}
    \item[(a)] Show that \( H \) is an abelian subgroup of \( G \).
    \item[(b)] Show that \( H \) is not finitely generated.
\end{itemize}

\textbf{Remarks.} \( GL_2(\mathbb{R}) \) denotes, as usual, the group (under matrix multiplication) of all \( 2 \times 2 \) invertible matrices with real entries (elements). \textit{Abelian} means commutative. A group is \textit{finitely generated} if there are a finite number of elements of the group such that every other element of the group can be obtained from these elements using the group operation.

\textbf{Solution.}
\begin{itemize}
    \item[(a)] All of the matrices in \( G \) are of the form
    \[
    \begin{bmatrix}
    * & * \\
    0 & *
    \end{bmatrix}.
    \]
    So all of the matrices in \( H \) are of the form
    \[
    M(x) = \begin{bmatrix}
    1 & x \\
    0 & 1
    \end{bmatrix},
    \]
    so they commute. Since \( M(x)^{-1} = M(-x) \), \( H \) is a subgroup of \( G \).

    \item[(b)] A generator of \( H \) can only be of the form \( M(x) \), where \( x \) is a binary rational, i.e., \( x = \frac{p}{2^n} \) with integer \( p \) and non-negative integer \( n \). In \( H \) it holds
    \[
    M(x)M(y) = M(x + y)
    \]
    and
    \[
    M(x)M(y)^{-1} = M(x - y).
    \]
    The matrices of the form \( M\left(\frac{1}{2^n}\right) \) are in \( H \) for all \( n \in \mathbb{N} \). With only finite number of generators all of them cannot be achieved.
\end{itemize}

\textbf{Problem 4.} (20 points) \\
Let \( B \) be a bounded closed convex symmetric (with respect to the origin)
set in \( \mathbb{R}^2 \) with boundary the curve \( \Gamma \). Let \( B \)
have the property that the ellipse of maximal area contained in \( B \)
is the disc \( D \) of radius 1 centered at the origin with boundary
the circle \( C \). Prove that \( A \cap \Gamma \neq \emptyset \)
for any arc \( A \) of \( C \) of length \( l(A) \geq \frac{\pi}{2} \).

\textbf{Solution.} Assume the contrary -- there is an arc \( A \subseteq C \) with length \( l(A) = \frac{\pi}{2} \) such that \( A \cap B/\Gamma \). Without loss of generality we may assume that the ends of \( A \) are \( M = (1/\sqrt{2}, 1/\sqrt{2}) \), \( N = (1/\sqrt{2}, -1/\sqrt{2}) \). \( A \) is compact and \( \Gamma \) is closed. From \( A \cap \Gamma = \emptyset \) we get \( \delta > 0 \) such that \( \text{dist}(x, y) > \delta \) for every \( x \in A \), \( y \in \Gamma \).

Given \( \varepsilon > 0 \) with \( E_{\varepsilon} \) we denote the ellipse with boundary: \( \frac{x^2}{(1 + \varepsilon)^2} + \frac{y^2}{b^2} = 1 \), such that \( M, N \in E_{\varepsilon} \). Since \( M \in E_{\varepsilon} \) we get

\[ b^2 = \frac{(1 + \varepsilon)^2}{2(1 + \varepsilon)^2 - 1}. \]

Then we have

\[ \text{area} E_{\varepsilon} = \pi \frac{(1 + \varepsilon)^2}{\sqrt{2(1 + \varepsilon)^2 - 1}} > \pi = \text{area} D. \]

In view of the hypotheses, \( E_{\varepsilon} \nsubseteq B \) for every \( \varepsilon > 0 \). Let \( S = \{(x, y) \in \mathbb{R}^2 : |x| > |y|\} \). From \( E_{\varepsilon} \subseteq S \subseteq D \subseteq B \) it follows that \( E_{\varepsilon} \nsubseteq B \subseteq S \). Taking \( \varepsilon < \delta \) we get that

\[ \emptyset \neq E_{\varepsilon} \backslash B \subseteq E_{\varepsilon} \cap S \subseteq D_{1+\varepsilon} \cap S \subseteq B \]

-- a contradiction (we use the notation \( D_t = \{(x, y) \in \mathbb{R}^2 : x^2 + y^2 \leq t^2\} \)).

\textbf{Remark.} The ellipse with maximal area is well known as John’s ellipse. Any coincidence with the President of the Jury is accidental.

\textbf{Problem 5.} (20 points)
\begin{itemize}
    \item[(i)] Prove that
    \[
    \lim_{x \to +\infty} \sum_{n=1}^{\infty} \frac{nx}{(n^2 + x)^2} = \frac{1}{2}.
    \]
    \item[(ii)] Prove that there is a positive constant \( c \) such that for every \( x \in [1, \infty) \) we have
    \[
    \left| \sum_{n=1}^{\infty} \frac{nx}{(n^2 + x)^2} - \frac{1}{2} \right| \leq \frac{c}{x}.
    \]
\end{itemize}

\textbf{Solution.}
\begin{itemize}
    \item[(i)] Set $f(t) = \frac{t}{(1 + t^2)^2}$, $h = \frac{1}{\sqrt{x}}$. Then
    \[
    \sum_{n=1}^{\infty} \frac{nx}{(n^2 + x)^2} = h \sum_{n=1}^{\infty} f(nh) \xrightarrow[h \to 0]{}
    \int_{0}^{\infty} f(t)dt = \frac{1}{2}.
    \]
    The convergence holds since $h \sum_{n=1}^{\infty} f(nh)$ is a Riemann sum of the integral $\int_{0}^{\infty} f(t)dt$. There are no problems with the infinite domain because $f$ is integrable and $f \to 0$ for $x \to \infty$ (thus $h \sum_{n=N}^{\infty} f(nh) \geq \int_{nN}^{\infty} f(t) dt \geq h \sum_{n=N+1}^{\infty} f(nh)$).
    \item[(ii)] We have
    \[
    (1) \quad \left| \sum_{n=1}^{\infty} \frac{nx}{(n^2 + x)^2} - \frac{1}{2} \right| =
    \left| \sum_{n=1}^{\infty} \left( hf(nh) - \int_{nh-\frac{h}{2}}^{nh+\frac{h}{2}} f(t)dt \right) - \int_{0}^{\frac{h}{2}} f(t)dt \right|
    \]
    \[
    \leq \sum_{n=1}^{\infty} \left| hf(nh) - \int_{nh-\frac{h}{2}}^{nh+\frac{h}{2}} f(t)dt \right| + \int_{0}^{\frac{h}{2}} f(t)dt
    \]
    Using twice integration by parts one has
    \[
    (2) \quad 2bg(a) - \int_{a-b}^{a+b} g(t)dt = -\frac{1}{2} \int_{0}^{b} (b-t)^2(g''(a + t) + g''(a - t))dt
    \]
    for every \( g \in C^2[a - b, a + b] \). Using \( f(0) = 0 \), \( f \in C^2[0, h/2] \) one gets
    \[
    (3) \quad \int_{0}^{h/2} f(t)dt = O(h^2).
    \]
    From (1), (2) and (3) we get
    \[
    \left| \sum_{n=1}^{\infty} \frac{nx}{(n^2 + x)^2} - \frac{1}{2} \right| \leq \sum_{n=1}^{\infty} h^2 \left| \int_{nh-\frac{h}{2}}^{nh+\frac{h}{2}} f''(t)dt \right| + O(h^2) =
    \]
    \[
    = h^2 \int_{h/2}^{\infty} |f''(t)|dt + O(h^2) = O(h^2) = O(x^{-1}).
    \]
\end{itemize}


\textbf{Problem 6. (Carleman’s inequality)} (25 points)

(i) Prove that for every sequence \(\{a_n\}_{n=1}^\infty\), such that \(a_n > 0\), \(n = 1,2,\ldots\) and \(\sum_{n=1}^{\infty} a_n < \infty\), we have

\[
\sum_{n=1}^{\infty} (a_1 a_2 \ldots a_n)^{1/n} < e \sum_{n=1}^{\infty} a_n,
\]

where \( e \) is the natural log base.

(ii) Prove that for every \( \varepsilon > 0 \) there exists a sequence \(\{a_n\}_{n=1}^\infty\), such that \(a_n > 0\), \(n = 1,2,\ldots\), \(\sum_{n=1}^{\infty} a_n < \infty\) and

\[
\sum_{n=1}^{\infty} (a_1 a_2 \ldots a_n)^{1/n} > (e - \varepsilon) \sum_{n=1}^{\infty} a_n.
\]

\textbf{Solution.}
\begin{enumerate}
    \item[(i)] Put for \( n \in \mathbb{N} \)
    \begin{equation}
        c_n = (n + 1)^n / n^{n-1}.
    \end{equation}
    Observe that \( c_1c_2 \cdot \ldots \cdot c_n = (n + 1)^n \). Hence, for \( n \in \mathbb{N} \),
    \begin{align*}
        (a_1a_2 \cdot \ldots \cdot a_n)^{1/n} &= (a_1c_1a_2c_2 \cdot \ldots \cdot a_nc_n)^{1/n} / (n + 1) \\
        &\leq (a_1c_1 + \ldots + a_nc_n)/n(n + 1).
    \end{align*}
    Consequently,
    \begin{equation}
        \sum_{n=1}^{\infty} (a_1a_2 \cdot \ldots \cdot a_n)^{1/n} \leq \sum_{n=1}^{\infty} a_nc_n \left( \sum_{m=n}^{\infty} (m(m + 1))^{-1} \right).
    \end{equation}
    Since
    \[
        \sum_{m=n}^{\infty} (m(m + 1))^{-1} = \sum_{m=n}^{\infty} \left( \frac{1}{m} - \frac{1}{m + 1} \right) = \frac{1}{n},
    \]
    we have
    \[
        \sum_{n=1}^{\infty} a_nc_n \left( \sum_{m=n}^{\infty} (m(m + 1))^{-1} \right) = \sum_{n=1}^{\infty} a_nc_n / n = \sum_{n=1}^{\infty} a_n((n + 1)/n)^n < e \sum_{n=1}^{\infty} a_n
    \]
(by (1)). Combining the last inequality with (2) we get the result.

(ii) Set \(a_n = n^{-1} \ln(n+1)^{-n}\) for \(n = 1,2,...,N\) and \(a_n = 2^{-n}\) for \(n > N\), where \(N\) will be chosen later. Then

\begin{equation}
    (a_1 \cdot \ldots \cdot a_n)^{1/n} = \frac{1}{n+1}
\end{equation}

for \(n \leq N\). Let \(K = K(\varepsilon)\) be such that

\begin{equation}
    \left( \frac{n+1}{n} \right)^n > e - \frac{\varepsilon}{2} \text{ for } n > K.
\end{equation}

Choose \(N\) from the condition

\begin{equation}
    \sum_{n=1}^{K} a_n + \sum_{n=N+1}^{\infty} 2^{-n} \leq \frac{\varepsilon}{(2e - \varepsilon)(e - \varepsilon)} \sum_{n=K+1}^{N} \frac{1}{n},
\end{equation}

which is always possible because the harmonic series diverges. Using (3), (4) and (5) we have

\[
    \sum_{n=1}^{\infty} a_n = \sum_{n=1}^{K} a_n + \sum_{n=N+1}^{\infty} 2^{-n} + \sum_{n=K+1}^{N} \frac{1}{n} \left( \frac{n}{n+1} \right)^n <
\]

\[
    < \frac{\varepsilon}{(2e - \varepsilon)(e - \varepsilon)} \sum_{n=K+1}^{N} \frac{1}{n} + \left( e - \frac{\varepsilon}{2} \right)^{-1} \sum_{n=K+1}^{N} \frac{1}{n} =
\]

\[
    = e - \varepsilon \sum_{n=K+1}^{N} \frac{1}{n} - e \sum_{n=K+1}^{N} \frac{1}{n} = e - \varepsilon - \sum_{n=1}^{\infty} (a_1 \cdot \ldots \cdot a_n)^{1/n}.
\]

\end{enumerate}

\end{document}
