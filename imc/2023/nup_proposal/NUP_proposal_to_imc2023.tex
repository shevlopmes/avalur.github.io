\documentclass[a4paper,12pt]{article}
\usepackage{amssymb,amsfonts,amsmath}
\usepackage[english]{babel}
\usepackage{latexsym}
\usepackage{epsfig}

% =======================================================================
% Margins --- save forests
\NeedsTeXFormat{LaTeX2e}
\oddsidemargin -10 pt       %   Left margin on odd-numbered pages.
\evensidemargin 10 pt       %   Left margin on even-numbered pages.
\marginparwidth 1 in        %   Width of marginal notes.
\oddsidemargin -1.5 true cm %   Note that \oddsidemargin = \evensidemargin
\evensidemargin -1.5 true cm
\marginparwidth 0.75 in
\textwidth 7.5 true in % Width of text line.
\textheight 26.0 true cm
\topmargin -2.5 true cm

% =======================================================================
% Environments problem, solution, remark

\newcount\probcnt
\newenvironment{problem}[1]{%
  \global\advance\probcnt1%
  \goodbreak\medskip\par\noindent\textbf{Problem~\the\probcnt%
    \if{#1}\empty\else~(#1)\fi.}~}%
{%
  \goodbreak
}

\newenvironment{solution}[1][]{%
  \goodbreak\smallskip\par\noindent\textbf{Solution{\if#1\empty\else~#1\fi}.}~}%
{%
  \goodbreak
}

\newenvironment{remark}[1][]{
  \goodbreak\smallskip\par
  \small
  \noindent\textbf{Remark{\if#1\empty\else~#1\fi}.}~%
}{%
  \goodbreak
  \normalsize
}

\newenvironment{lemma}[1][]{
  \goodbreak\smallskip\par
  \noindent\textit{Lemma{\if#1\empty\else~#1\fi}.}~%
}{%
  \goodbreak\smallskip
}

\newenvironment{proof}[1][]{
  \goodbreak\par
  \noindent\textit{Proof{\if#1\empty\else~#1\fi}.}~%
}{%
  \goodbreak\smallskip
}


% =======================================================================
%%% Some common macros
\newcommand{\RR}{{\mathbb{R}}}
\newcommand{\ZZ}{{\mathbb{Z}}}
\newcommand{\CC}{{\mathbb{C}}}
\newcommand{\QQ}{{\mathbb{Q}}}
\newcommand{\NN}{{\mathbb{N}}}
\newcommand{\tr}{\rm tr}
\newcommand{\ds}{\displaystyle}
\newcommand{\dx}{\mathrm{d}x}
\newcommand{\dy}{\mathrm{d}y}
\newcommand{\dz}{\mathrm{d}z}
\newcommand{\dt}{\mathrm{d}t}
\newcommand{\du}{\mathrm{d}u}
\newcommand{\GL}{\operatorname{GL}}
\newcommand{\rk}{\operatorname{rk}}

% == TITLES ==========================================================
\begin{document}
\begin{center}
  {\Large\textbf{Proposed problems for the IMC 2023}}
\end{center}

% =======================================================================
%%% DOCUMENT_BEGIN
% Do not change the document above this point
% =======================================================================

% =======================================================================
% My macros
% =======================================================================
% Insert your macros here

% \def\MyMacro{...}


% =======================================================================

\begin{problem}{Alex Avdiushenko, Neapolis University Paphos, Cyprus}
  Find all non-constant functions that have a continuous 
  second derivative and for which the equality 
  \(f(7x+1) = 49f(x)\) holds for all \(x\).

  \begin{remark}
    A simple problem, suitable for the place of the first.
  \end{remark}
\end{problem}

% -----------------------------------------------------------------------

\begin{solution}
  
  Differentiating the equation twice, we get 
  \(f''(7x+1) = f''(x)\), or 
  \(f''(x) = f''\left(\frac{x - 1}{7}\right)\) — 1 point.
  
  Take an arbitrary \(x = a\) and construct a sequence 
  according to the rule 
  \(a_0 = a\), \(a_{k+1} = \frac{a_k - 1}{7}\). 
  Then the values of \(f''\) at all points of this sequence are equal. 
  The limit of this sequence 
  is \(-\frac{1}{6}\), since \(\left|a_{k+1} + \frac{1}{6}\right| = \frac{1}{7}\left|a_k + \frac{1}{6}\right|\) — 5 points.
  
  Due to continuity, the values of \(f''\) at all points of this sequence 
  are equal to \(f''\left(-\frac{1}{6}\right)\), which means that \(f''(x)\) 
  is a constant — 2 points.
  
  Then \(f(x) = c_2x^2 + c_1x + c\). Substituting this expression into 
  the original equation, we get a system of equations, 
  from which we find \(c_2 = 36c\), \(c_1 = 12c\), and \(f(x) = (6x + 1)^2c\) — 2 points.
  
\end{solution}

% =======================================================================

\begin{problem}{Alex Avdiushenko, Neapolis University Paphos, Cyprus}
  Given the matrix 
  \[
  A = 
  \begin{pmatrix}
  2 & 3 \\
  2 & 4
  \end{pmatrix}
  \]
  It is allowed to multiply or divide any row (column) element-wise by another row (respectively, column). 
  Is it possible to obtain the following matrix $B$ after several such operations?
  \[
  B = 
  \begin{pmatrix}
  2 & 4 \\
  2 & 3
  \end{pmatrix}
  \]
  
  \begin{remark}
    A simple problem, suitable for the place of the first or may be second one.
  \end{remark}
\end{problem}

% -----------------------------------------------------------------------

\begin{solution}
  Answer: No, it's not possible.
  
  Let's represent all numbers in the matrix as powers of 2 (or any other number). 
  Then the specified operations are reduced to adding or subtracting 
  the corresponding exponents — 4 points. 
  
  Let's construct a matrix from the exponents: 
  \[
  C = 
  \begin{pmatrix}
  1 & \log_2 3 \\
  1 & 2
  \end{pmatrix}
  \]
  
  The determinant of this matrix does not change under the given operations — 4 points. 
  
  But it is not equal to $0$ and is opposite to the determinant of 
  the exponents of the original matrix — 2 points.
\end{solution}

% =======================================================================

\begin{problem}{Alex Avdiushenko, Neapolis University Paphos, Cyprus}
  Quadratic polynomials \(p_1(x) = a_1x^2 + b_1x + c_1\) and 
  \(p_2(x) = a_2x^2 + b_2x + c_2\) with integer coefficients do not have common roots. 
  For each \(n \in \mathbb{N}\), let \(d_n = \gcd \left(f(n)g(n)\right)\).
  Prove that the sequence \(\{d_n\}_{n=1}^\infty\) is bounded.

  \begin{remark}
    A more difficult problem, suitable for the place of the second or third.
  \end{remark}
\end{problem}

% -----------------------------------------------------------------------

\begin{solution}
  Let \(d_n = \left(f(n)g(n)\right)\). 
  Then \(d_n|f(n)\) and \(d_n|g(n)\) for any \(n \in \mathbb{N}\), 
  where $|$ denotes divisibility. 

  Therefore, \(d_n | a_1p_2(n)-a_2p_1(n) = \gamma n + \beta\), where 
  $\gamma = a_1b_2-a_2b_1, \beta = a_1c_2-a_2c_1$.
  
  This means that \( \gamma n \equiv -\beta \mod d_n\) — 4 points.

  Consider $\gamma^2 p_i(n) \equiv 0 \mod d_n \Rightarrow a_i\beta^2 - b_i\gamma\beta + c_i \gamma^2 \equiv 0 \mod d_n$ — 4 points.

  In the case of an unbounded sequence \(\{d_n\}\), it follows from here 
  that \(a_i\beta^2 - b_i\gamma\beta + c_i \gamma^2 = 0,\ i=1,2 \), 
  which means \(f(n)\) and \(g(n)\) have a common root \(x = -\frac{\beta}{\gamma}\) — 2 points.
\end{solution}

% =======================================================================

\begin{problem}{Alex Avdiushenko, Neapolis University Paphos, Cyprus}
  Let \(P\) and \(Q\) be square matrices such that \(P^k = Q^l = 0\), 
  where \(k, l \in \NN\), and \(0\) is the zero matrix. 
  Prove that if \(P\cdot Q = Q\cdot P \), then the matrix 
  \(I + \lambda P + \mu Q\) is invertible. 
  Here, \(\lambda, \mu \in \RR \), \(I\) — the identity matrix.

  \begin{remark}
    A more difficult problem, suitable for the place of the second or third.
  \end{remark}
\end{problem}

% -----------------------------------------------------------------------

\begin{solution}
  Let \(S = \lambda P + \mu Q\). 
  Then \( S^{k+l-1} = \sum\limits_{i=0}^{k+l-1} 
  \begin{pmatrix}
    k+l-1 \\ i
  \end{pmatrix}
  \lambda^iP^i \mu^{k+l-1-i}Q^{k+l-1-i} = 0 \),
  since either the power of \(P\) is not less than \(k\), 
  or the power of \(Q\) is not less than \(l\). 
  Further, by expanding the brackets, it is checked that 
  \(\left(I + S\right) \left(I - S + S^2 - \dots + (-1)^{k+l-2}S^{k+l-2}\right)\), 
  i.e. the required matrix is invertible.
\end{solution}

% =======================================================================
% Do not change the document below this point
%%% DOCUMENT_END
% =======================================================================

\end{document}

